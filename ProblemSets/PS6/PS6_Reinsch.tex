\documentclass{article}
\usepackage[utf8]{inputenc}

\title{PS6_Reinsch}
\author{Emma Reinsch}
\date{March 2021}

\begin{document}



\section{Data Cleaning Steps}
Visualization 1: I obtained data from the FRED website. I obtained income overtime for African Americans in the United States. I filtered the data to only include years ranging from 1990 to March 2021. I also transformed the data into a data frame.


Visualization 2: I obtained my second form of data from the FRED website. I obtained income overtime for White/Caucasian individuals in the United States. I filtered the data to only include years ranging from 1990 to March 2021. Once again, I transformed the data set into a data.frame.


Visualization 3: I obtained my third form of data from the IPUMS USA website. I used American Community Survey data (ACS) and gathered several variables. My objective is to regress African Americans on income wage to see the residual wage gap. I mutated the "RACBLK" variable coded 1 for no (not black) and 2 for yes into a dummy variable named "black". After filtering income wage to include values from $10000$ dollars to $300000$ dollars and total income from $50000$ dollars to $350000$ dollars, I obtained the count of African American individuals as opposed to non-African American individuals. I intend to repeat the same for the "RACWHT" variable (coded 1 for non-white and 2 for white). 

I also took the natural logarithm of the income variable. I intended to create a graph with race as the independent variable and log(incomewage) as the dependent variable, but ran into some issues. I will be making further adjustments to the data set.


\end{document}
